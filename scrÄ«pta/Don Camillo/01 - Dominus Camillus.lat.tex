\documentclass[a4paper, 12pt]{article}

\usepackage[hidelinks, bookmarksopen=true, bookmarks=true]{hyperref}
\usepackage{bookmark}

\usepackage[latin]{babel}
\usepackage{comment}
\usepackage{csquotes}
\usepackage{extsizes}
\usepackage{fancyhdr}
\usepackage[T1]{fontenc}
\usepackage[margin=3cm]{geometry}
\usepackage[utf8x]{inputenc}
\usepackage{letltxmacro}
\usepackage{lineno}
\usepackage{tempora}
\usepackage{microtype}
\usepackage{setspace}
\usepackage{titlesec}
\usepackage{titling}
\usepackage{xcolor}
\usepackage{xunicode}

% font
\usepackage[lining]{ebgaramond}
\newcommand{\red}[1]{{\color{red}  {#1}}}

\newcommand{\MyTitle}{\textbf{Dominus Camillus}}

% document settings
\title{
	\MyTitle
}
\author{\textit{scrīpsit} Giovannino Guareschi \and \textit{vertit} Michele Palese}
\date{}

% headers and footers
\pagestyle{fancy}
\fancyhf{}
\lhead{\MyTitle}
\rhead{\footnotesize\theauthor}
\cfoot{\thepage}

% section numbering
\renewcommand{\thesection}{}%\Roman{section}} 
\renewcommand{\thesubsection}{\thesection.\Roman{subsection}}
\titleformat*{\section}{\normalsize\bfseries}

\modulolinenumbers[5]
% line stretch
%\renewcommand{\baselinestretch}{1.73}

% footnotes size & font
\renewcommand\linenumberfont{\normalfont\footnotesize}

\interfootnotelinepenalty=10000 % do not split across pages

\newcommand{\sta}{
	%\bigskip
	\begin{center}
		\Large * * *
	\end{center}
	%\bigskip
}

\begin{document}
	
	\maketitle
	%\break
	
	Dominus Camillus, archipresbyter vīcī Ponterotto, erat vir summae benignitātis.
	Sed erat ūnus ex eōrum quī apertē prŏfitentur et, tum cum rēs perplexae sordidaeque accidērunt, in quibus senēs locuplēs et virginēs immixtī erant, dominus Camillus in Missā ōrātiōnem infīnītam comiter incēpit, sed posteā, id temporis cum sānē prīmō ōrdine ūnum ex turpium vīdit, modō victō et ōrātiōne abruptā, pannum super caput Iēsū crucī affīxī altāris maiōris iēcit nē audīret et, manūs in lateribus pōnēns, suō modō ōrātiōnem ad fīnem perdūxit et tam tonāns erat vōx quae ex ōre illīus hominis immāni corporis magnitūdine, et tam magna dīcēbat, ut tēctum parvae ecclēsiae tremeret.
	
	Dominus Camillus scīlicet, tempore ēligendī adventō, tam apertē dē prīncipibus sinistrārum partium locī ōrātiōnibus dīxit ut, ad vesperem, vixdum satis certā lūce, dum reveniēbat ad parochī domum, magnus prāvusque homō palliō amictus ex saepe exsiliēns ā tergō adorīrētur et, tempus in quō dominus Camillus impediēbātur ā birotā cuius manūbriō suspēnsa erat sarcina in quā septuāgintā ōva inerant nōn dīmittēns, validē tunderet eum pālō, ēvolāns posteā ē cōnspectū ut ā terrā absorptus.
	
	Dominus Camillus nihil nēminī dīxit.
	In parochī domum adventus et ōvīs in tūtō positīs, īvit in ecclēsiam ad Iēsum cōnsulendum, sīcut solēbat cum dubitāret.
	
	«Quid faciam?» quaesīvit dominus Camillus.
	
	«Pēnicillō dorsum tuum aquā oleō mixtā illine et tacētō» respondit eī Iēsūs ab altāris summitāte.
	«Iniūriās īnferentibus dōnandum est. Hoc est praeceptum.»
	
	«Concēdō» opposuit dominus Camillus.
	«Sed haec sunt verbera, nōn iniūriae.»
	
	«Et quid hoc significat?» eī susurrāvit Iēsūs.
	«Num vīs iniūriās corporis iniūriīs animae acerbiōrēs esse?»
	
	«Cōnsentiō, Domine.
	Sed Tū illōs mē, quī sum minister Tuus, verberantēs Tibi iniūriās intulisse meminisse dēbē.
	Hoc agō prō Tē nec prō mē.»
	
	«Num ego tē magis minister Deī eram?
	Et num dōnāvī illīs mē in cruce fīgentibus?»
	
	«Tū es difficilis et contumāx» conclūsit dominus Camillus.
	«Semper rēctē dīcis Tū.
	Fīat voluntās Tua.
	Dōnābimus.
	Mementō tamen sī illī meō silentiō ēlātī capitem meum frangent culpam Tuam esse.
	Possum Tibi aliquot locōs Veteris Testāmentī commemorāre...»
	
	«Domine Camille, mihi vīs dē Vetere Testāmentō loquī!
	Ea quae ad cētera pertinent mē auctōre sunt.
	Sed, inter nōs, aliquot ictibus dignus es, nē rem pūblicam agās domī meā.»
	
	Dominus Camillus dōnāvit.
	
	\sta
	
\end{document}
