\documentclass[a4paper, 12pt]{article}

\usepackage[hidelinks, bookmarksopen=true, bookmarks=true]{hyperref}
\usepackage{bookmark}

\usepackage[latin]{babel}
\usepackage{comment}
\usepackage{csquotes}
\usepackage{extsizes}
\usepackage{fancyhdr}
\usepackage[T1]{fontenc}
\usepackage[margin=3cm]{geometry}
\usepackage[utf8x]{inputenc}
\usepackage{letltxmacro}
\usepackage{lineno}
\usepackage{tempora}
\usepackage{microtype}
\usepackage{setspace}
\usepackage{titlesec}
\usepackage{titling}
\usepackage{xcolor}
\usepackage{xunicode}

% font
\usepackage[lining]{ebgaramond}
\newcommand{\red}[1]{{\color{red}  {#1}}}

\newcommand{\MyTitle}{\textbf{Dominus Camillus}}

% document settings
\title{
	\MyTitle
}
\author{\textit{scrīpsit} Giovannino Guareschi \and \textit{vertit} Michele Palese}
\date{}

% headers and footers
\pagestyle{fancy}
\fancyhf{}
\lhead{\MyTitle}
\rhead{\footnotesize\theauthor}
\cfoot{\thepage}

% section numbering
\renewcommand{\thesection}{}%\Roman{section}} 
\renewcommand{\thesubsection}{\thesection.\Roman{subsection}}
\titleformat*{\section}{\normalsize\bfseries}

\modulolinenumbers[5]
% line stretch
%\renewcommand{\baselinestretch}{1.73}

%footnotes enabling and disabling
\newcommand{\sfootnote}[1]{{\EnableFootNotes\footnote{\textit{#1}}}}
\LetLtxMacro\Oldfootnote\footnote
\newcommand{\EnableFootNotes}{%
	\LetLtxMacro\footnote\Oldfootnote%
}
\newcommand{\DisableFootNotes}{%
	\renewcommand{\footnote}[2][]{\relax}
}
% footnotes size & font
\renewcommand\linenumberfont{\normalfont\footnotesize}

\interfootnotelinepenalty=10000 % do not split across pages

\begin{document}
	
	\maketitle
	%\break
	
	\DisableFootNotes
	
	\resetlinenumber
	
	Dominus Camillus, archipresbyter vīcī Ponterotto, erat vir summae benignitātis.
	Sed erat ūnus ex eōrum quī apertē prŏfitentur et, tum cum rēs perplexae sordidaeque accidērunt, in quibus senēs locuplēs et virginēs immixtī erant, dominus Camillus in Missā ōrātiōnem infīnītam comiter incēpit, sed posteā, id temporis cum sānē prīmō ōrdine ūnum ex turpium vīdit, modō victō et ōrātiōne abruptā, pannum super caput Iēsū crucī affīxī altāris maiōris iēcit nē audīret et, manūs in lateribus pōnēns, suō modō ōrātiōnem ad fīnem perdūxit et tam tonāns erat vōx quae ex ōre illīus hominis immāni corporis magnitūdine, et tam magnā dīcēbat, ut tēctum parvae ecclēsiae tremeret.
	
	Dominus Camillus scīlicet, tempore ēligendī adventō, tam apertē dē locī prīncipibus sinistrārum partium ōrātiōnibus dīxit ut, ad vesperem, vixdum satis certā lūce, dum reveniēbat ad parochī domum,
	
\end{document}
