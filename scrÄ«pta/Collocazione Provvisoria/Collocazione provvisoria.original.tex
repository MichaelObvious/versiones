\documentclass[a4paper, 12pt]{article}

\usepackage{array}
\usepackage[italian]{babel}
\usepackage{xcolor}
\usepackage{comment}
\usepackage[lining]{ebgaramond}
\usepackage{fancyhdr}
\usepackage[margin=2cm]{geometry}
\usepackage{graphicx}
\usepackage{longtable}
\usepackage{microtype}
\usepackage{titlesec}

\newcommand{\MyTitle}{\textbf{Collocazione Provvisoria}}

\title{
	\Huge{\MyTitle}
}
\author{don Tonino Bello}
\date{}

\pagestyle{fancy}
\fancyhf{}
\lhead{\MyTitle}
\rhead{\footnotesize\theauthor}
\cfoot{\thepage}

\begin{document}
	\maketitle
	
	Nel Duomo vecchio di Molfetta (Bari) c'è un grande crocifisso di terracotta. Il parroco in attesa di sistemarlo definitivamente l'ha addossato alla parete della sagrestia, e vi ha apposto un cartoncino con la scritta: \textit{``Collocazione provvisoria''}.

    La scritta mi è parsa provvidenzialmente ispirata.
    \textit{``Collocazione provvisoria''}: penso che non ci sia formula migliore per \textbf{definire la croce}.
    \textbf{La mia, la tua croce, non solo quella di Cristo}.

    \textbf{Coraggio} allora, tu che soffri.
    Animo, tu che provi i morsi della solitudine.
    Abbi fiducia, tu che bevi il calice amaro dell'abbandono.
    Asciugati le lacrime, fratello, che sei stato pugnalato alle spalle da coloro che ritenevi tuoi amici.
    Non angosciarti, tu che per un tracollo improvviso vedi i tuoi progetti in frantumi, le tue fatiche distrutte.
    Non tirare i remi in barca, tu che sei stanco di lottare e hai accumulato delusioni a non finire.
    Non abbatterti, fratello povero, che non sei calcolato da nessuno.
    Non avvilirti, amico sfortunato, che nella vita hai visto partire tanti bastimenti, e tu sei rimasto sempre a terra.

    \textbf{Coraggio. La tua croce è sempre \textit{``collocazione provvisoria''}.}
    Il Calvario, dove essa è piantata, non è zona residenziale.
    Anche il Vangelo ci invita a considerare la provvisorietà della croce.
    C'è una frase immensa, che riassume la tragedia del creato al momento della morte di Cristo.
    \textit{``Da mezzogiorno fino alle tre del pomeriggio, si fece buio su tutta la terra''}.
    Forse è la frase più scura di tutta la Bibbia.
    Per me è una delle più luminose.
    Proprio per quelle riduzioni di orario che stringono, come due paletti invalicabili, il tempo in cui è concesso al buio di infierire sulla terra.

    Da mezzogiorno alle tre del pomeriggio.
    Solo allora è consentita la sosta sul Golgota.
    Al di fuori di quell'orario c'è divieto assoluto di parcheggio.
    Dopo tre ore, ci sarà la rimozione forzata di tutte le croci.

    \textbf{Coraggio}, fratello che soffri.
    C'è anche per te una deposizione della croce.
    C'è anche per te una pietà sovraumana.
    Ecco già una mano forata che schioda dal legno la tua.
    \textbf{Coraggio}.
    Mancano pochi istanti alle tre del pomeriggio.
    Tra poco, il buio cederà il posto alla luce, la terra riacquisterà i suoi colori verginali e il sole della Pasqua irromperà tra le nuvole in fuga.

	\bigskip
	
	\raggedleft \textit{--- don Tonino Bello} \hspace{1em}
\end{document}
