\documentclass[a4paper, 12pt]{article}

\setcounter{tocdepth}{2}

\usepackage[hidelinks, bookmarksopen=true, bookmarks=true]{hyperref}
\usepackage{bookmark}

\usepackage[italian]{babel}
\usepackage{comment}
\usepackage{csquotes}
\usepackage{extsizes}
\usepackage{fancyhdr}
\usepackage[T1, T2A]{fontenc}
\usepackage[utf8x]{inputenc}
\usepackage{letltxmacro}
\usepackage{lineno}
\usepackage{tempora}
\usepackage{microtype}
\usepackage{setspace}
\usepackage{titlesec}
\usepackage{titling}
\usepackage{xcolor}
\usepackage{xunicode}

% font
\usepackage[lining]{ebgaramond}

\newcommand{\MyTitle}{\textbf{Il bambino alla festa di Natale di Cristo}}

% document settings
\title{
	\MyTitle\\
	\large\textit{da ``Diario di uno scrittore'', gennaio 1876}
}
\author{\textit{testo di} Fëdor M. Dostoevskij \and \textit{traduzione a cura di} Michele Palese}
\date{}

% headers and footers
\pagestyle{fancy}
\fancyhf{}
\lhead{\MyTitle}
\rhead{\footnotesize{Fëdor M. Dostoevskij, \textit{trad. Michele Palese}}}
\cfoot{\thepage}

% section numbering
\renewcommand{\thesection}{}%\Roman{section}} 
\renewcommand{\thesubsection}{\thesection.\Roman{subsection}}
\titleformat*{\section}{\normalsize\bfseries}

\modulolinenumbers[5]
% line stretch
\renewcommand{\baselinestretch}{1}

% footnotes size & font
\renewcommand\linenumberfont{\normalfont\scriptsize}

\newcommand{\red}[1]{{\color{red}  {#1}}}

\interfootnotelinepenalty=10000 % do not split across pages

\begin{document}
	
	\maketitle
	\break
	
	\resetlinenumber
\begin{linenumbers}
	
	\section{Il bambino ``con la mano''}
	
	I bambini sono creature strane, appaiono nei sogni e immaginiamo di vederli. %TODO fix not really italian
	Prima della vigilia e alla vigilia di Natale stessa mi accadeva d'incontrare per la strada, ad un angolo noto, un bambino, che non aveva più di sette anni.
	Nonostante il freddo terribile era vestito quasi come d'estate, ma il suo collo era avvolto da un qualche straccio usurato, -- significa che qualcuno l'aveva attrezzato, mandandolo.
	Egli andava ``con la mano''; questo è un termine tecnico, significa -- chidere la carità.
	Tale termine l'hanno inventato questi stessi bambini.
	Ragazzini così, come lui, ce n'è una moltitudine, essi girano per la vostra strada e ululano qualcosa che hanno imparato a memoria; ma questo non ululava e parlava così innocentemente e come se non fosse ancora abituato e mi guardava speranzoso negli occhi, -- dev'essere che aveva appena cominciato la professione.
	Alle mie domande rispose, che ha una sorella, senza lavoro, malata.
	Magari, è anche vero, ma seppi solo dopo, che di questi ragazzi ce n'è una quantità innumerevole:
	Li mandano ``con la mano'', anche nel freddo più rigido, e se non avranno raccolto nulla, probabilmente li aspetteranno delle botte.
	Raccolte le monetine il bambino torna con le mani rosse, insensibili a causa del freddo in un qualche seminterrato, dove una banda di lazzaroni si ubriaca, di quelli che ``dopo aver smesso di lavorare la sera del sabato in vista della domenica, tornano al lavoro non prima del mercoledì sera''.
	Lì, nei seminterrati, si ubriacano con essi le loro mogli picchiate e affamate, e sempre lì gemono i neonati affamati.
	Vodka, sporcizia, e dissolutezza, e sopratutto, vodka.
	Con le monetine raccolte il ragazzo viene mandato immediatamente al \textit{kabak}\footnote{Il \textit{kabàk} è un luogo in cui vengono consumate e vendute bevande, simile ad un \textit{bar}.}, per tornare poi con del vino.
	Per divertimento versano pure a lui una \textit{kasushka}\footnote{La \textit{kasúshka} è un unità di misura per i liquidi russa ormai caduta in disuso a causa dell'introduzione del sistema metrico. Equivale a circa \textit{0.3 litri}.} in bocca e ridono, quando lui, senza fiato, cade a terra quasi privo di sensi
	\begin{center}\textit{
		...ed in bocca della disgustosa vodka\\
		crudelmente mi versava...}\footnote{cfr. ``Infanzia'', N. A. Nekrasov}
	\end{center}
	
	Quando cresce, lo sbattono in fretta in una qualche fabbrica, ma tutto ciò che  guadagna, è di nuovo obbligato a portarlo ai lazzaroni, e loro spenderanno tutto bevendo.
	Ma anche prima di andare in fabbrica questi bambini diventano dei completi criminali.
	Essi vagano per la città e conoscono certi luoghi in diversi seminterrati, nei quali ci si può intrufolare e dove si può pernottare inosservati.
	Uno di loro dormì per alcune notti di fila da un certo portinaio in una qualche cesta, ed egli non se n'accorse.
	Da soli, in questo modo, diventano dei ladruncoli.
	Il furto diventa un'ossessione anche per i bambini di otto anni, a volte anche senza alcuna coscienza della criminalità delle loro azioni.
	Alla fine sopportano tutto -- la fame, il freddo, le botte, -- per un unica ragione, per la libertà, e scappano dai loro lazzaroni per vagare da soli.
	Questi esseri selvaggi a volte non capiscono nulla, né dove vivono, né a quale nazione appartengano, né se ci sia Dio, né se ci sia un sovrano;
	di loro dicono certe cose, che a sentirle non sembrano vere, e, tuttavia, corrispondono alla realtà.
	
	\section{\MyTitle}
	
	Ma io sono un romanziere, e, a quanto pare, ho inventato una ``storia''.
	Perché scrivo: ``a quanto pare''? Io in fondo so, che l'ho inventata, ma continuo ad avere l'impressione, che da qualche parte e in un qualche tempo ciò sia accaduto, proprio ciò sia accaduto la sera della vigilia di Natale, in \textit{una qualche} grande città ed in un freddo terribile.
	
	\vspace{1em}
	
	Mi sembra di vedere un ragazzo in un seminterrato, ma ancora molto piccolo, di sei anni o anche meno.
	Questo bambino si era svegliato al mattino in un umido e freddo seminterrato.
	Era vestito di una camicetta e tremava.
	Il suo respiro usciva dalla sua bocca come vapore bianco, e lui, seduto in un angolo sopra ad un baule, per la noia espirava questo vapore dalla bocca e si divertiva a guardare come volava via.
	Lui però aveva molta fame.
	Dal mattino alcune volte si era avvicinato al tavolaccio, dove sopra ad un finissimo tappetino, con il capo appoggiato sopra ad un fagotto che le faceva da cuscino, giaceva la sua madre malata.
	Come sarà finita lì? Dev'essere che è arrivata con il suo bambino da un'altra città e di punto in bianco si è ammalata.
	La padrona di questi ``angolini'' fu prelevata due giorni prima dalla polizia;
	gli inquilini si erano dispersi per le feste, e rimaneva unicamente un lazzarone, che da un giorno giaceva ubriaco, come morto; non aveva aspettato le feste.
	In un altro angolo gemeva a causa dei reumatismi una vecchietta ottantenne, che fu un tempo una bambinaia, e ora moriva da sola, gemendo, ringhiando e urlando al bambino, tantoché lui incominciò ad aver paura di passare vicino a lei.
	Da bere aveva trovato nel vestibolo, ma non aveva trovato da nessuna parte delle croste di pane, e già per la decima volta andava a svegliare sua mamma.
	Gli venne paura infine, a stare al buio: da tempo ormai era cominciata la sera, ma i lampioni non venivano ancora accesi.
	Tastando il volto della mamma, si stupì, nel sentire che essa non si muove e che è diventata fredda quanto il muro.
	``Fa proprio freddo qua'', -- pensò lui, stette un attimo, dimenticando senza accorgersi la mano sulla spalla della defunta, poi soffiò sulle sue ditina, per scaldarle, e, dopo aver trovato il suo berrettino, lentamente, tastoni, uscì dal seminterrato.
	Sarebbe uscito anche prima, ma aveva paura del grande cane, che stava sopra, sulla scala, e che tutto il giorno era davanti alla porta dei vicini.
	Ma il cane non c'era più, e all'improvviso uscì in strada.
	
	Dio, che citta!
	Non aveva mai visto qualcosa del genere.
	Lì, da dove era venuto, di notte c'è un buio così nero, e c'è un solo lampione per illuminare tutta la strada.
	Le imposte delle piccole e basse case di legno vengono chiuse;
	per strada, appena comincia ad imbrunire -- non c'è nessuno, tutti si chiudono in casa, e ci sono solo interi branchi di cani che ululano, centinaia e migliaia di loro, che ululano e abbaiano per tutta la notte.
	Ma lì perlomeno era così caldo e gli davano da mangiare, qui invece -- Dio, se potesse mangiare!
	E che frastuono e che rumore qui, quanta luce e quante persone, cavalli e carrozze, e il freddo, il freddo!
	Del vapore gelato fluisce dai cavalli lanciati, dai loro musi che espirano calore;
	nonostante la neve soffice suonano contro il selciato i ferri di cavallo, e tutti si spingono, e, Signore, quanto vuole mangiare, anche solo un piccolo pezzetto, e quanto gli facevano male le ditina ora.
	Una guardia passò vicino e si voltò dall'altra parte, per non doversi accorgere del bambino.
	
	Ecco un'altra strada, -- oh, quanto è larga!
	Ecco, qui forse mi schiacciano;
	tutti urlano, corrono e vanno, e quanta luce, quanta luce!
	E cos'è questo?
	Uh, che grande vetro, e dietro al vetro c'è una stanza, e nella stanza c'è un albero alto fino al soffitto;
	è un abete, e sull abete quante luci, quante cartine dorate e quante mele, e intorno ci sono delle bambole, e dei piccoli cavallini;
	e per la stanza corrono dei bambini, elegnati, ben puliti, ridono e giocano, e mangiano, e bevono qualcosa.
	Questa bambina si è messa a ballare con un altro ragazzino, che bella bambina!
	C'è pure la musica, si sente attraverso il vetro.
	Il bambino guarda, si meraviglia, e ride, ma ormai gli fanno male anche le ditina dei piedi, mentre sulle mani sono diventate completamente rosse, non si piegano più e fa male anche solo muoverle.
	D'un tratto il bambino si ricordò che gli fanno male le ditina, pianse e corse più avanti, ed ecco di nuovo vede attraverso un altro vetro un'altra stanza, anche lì ci sono degli alberi, e sul tavolo ci sono dei \textit{pirogi}\footnote{I pirogi (pron. \textit{piroghì}) sono un piatto tipico della cucina polacca ed est-europea, sono delle paste ripiene simili ai ravioli.} , di tutti i tipi -- farciti di mandorle, rossi, gialli, e siedono li quattro ricche signore, e a chi arriva, danno pirogi, e la porta si apre continuamente, molti signori entrano dalla strada.
	Il bambino si avvicinò, e improvvisamente aprì la porta ed entrò.
	Uh, quanto gridarono e agitarono le braccia!
	Una delle signore si avvicino velocemente e gli infilo in mano una monetina e gli aprì la porta.
	Quanto si spaventò!
	La monetina gli scivolò dalle mani e rotolò, risuonando, sugli scalini: non poteva infatti piegare le sue rosse ditina e trattenerla.
	Uscì fuori, e corse veloce-veloce, senza sapere dove stesse andando.
	Gli viene voglia di piangere di nuovo, e ha paura, e corre, corre e soffia sulle sue ditina.
	Di colpo il terrore lo sommerse, improvvisamente si sentì così solo e spaventato, e d'un tratto, Dio!
	E ora cos'è questo?
	C'è una folla di persone meravigliate:
	Nella finestra dietro al vetro ci sono tre bambole, piccoline, vestite di verde e di rosso, e sembrano proprio vive!
	Uno è un vecchietto seduto che pare suonare un grosso violino, gli altri due stanno in piedi e suonano anche loro due piccoli violini, muovono la testa a ritmo, si guardano l'un l'altro, e le loro labbra si muovono, parlano, certamente parlano, -- però attraverso il vetro non si sente.
	All'inizio il bambino pensava che fossero vive, e quando comprese che erano degli automi, improvvisamente rise forte.
	Non aveva mai visto delle bambole così e non sapeva che esistessero!
	Ha voglia di piangere, ma le bambole sono così divertenti.
	D'un tratto gli sembrò che qualcuno da dietro l'avesse afferrato per la camicetta:
	Un perfido ragazzaccio stava dietro di lui e d'un tratto lo colpì forte in testa, strappò il berretto, facendogli uno sgambetto.
	Il bambino ruzzolò a terra, intorno urlarono, lui rimase inebetito, poi balzò in piedi e cominciò a correre forte, improvvisamente finì da qualche parte, nemmeno lui sapeva dove, s'infilò in un portone e si sedette dietro ad un mucchio di legna: ``Qui non mi troveranno, e poi è pure buio''.
	
	Sedeva rattrappito, e non riusciva a respirare per la paura e d'improvviso, veramente d'improvviso, si sentì così bene:
	le manine e i piedini smisero di fargli male e venne un tepore, come sopra ad una stufa; poi rabbrividì:
	ah, ma si era addormentato!
	Come è piacevole addormentarsi qui: ``Rimarrò qui un po' e poi tornerò a guardare le bambole, -- pensò il bambino e sorrise, -- sembrano davvero vive!..''
	Ad un certo punto gli parve che sopra di lui sua mamma avesse cominciato a cantare una canzone.
	``Mamma, sto dormendo, ah, com'è bello dormire qui!''
	
	-- Vieni da me, andiamo all'albero, piccino, -- sussurrò sopra di lui una voce calma.
	
	Lui pensò che fosse ancora sua mamma, ma no, non è lei;
	Non riusciva a vedere chi l'aveva chiamato, ma qualcuno si chinò su di lui e lo abbracciò nell'oscurità, e lui gli tese la mano e... e d'un tratto, -- quanta luce! Che bell'albero di Natale!
	Ma no, non è un albero di Natale, non ne aveva mai visto uno così!
	Dov'è adesso: tutto brilla, tutto splende e intorno è pieno di bambole, -- anzi no, sono tutti bimbi e bimbe, solo così luminosi, girano introno a lui, volano, lo baciano, lo prendono, lo portano con loro, e lui stesso vola, e vede: sua mamma lo guarda e ride felice.
	
	-- Mamma! Mamma! Ah, come si sta bene qui, mamma! -- le grida il bimbo, e di nuovo si scambia baci con gli altri bimbi, e vuole raccontar loro al più presto degli automi che ha visto dietro il vetro, -- Chi siete, bimbi? Chi siete, bimbe? -- chiede lui, ridendo pieno d'amore per loro.
	
	-- Questo è l' ``albero di Natale di Gesù'',  -- gli rispondono. -- Da Gesù c'è sempre in questo giorno un albero di Natale per i piccoli bambini che non ne hanno uno proprio... --
	E venne a sapere che tutti i bimbi e le bimbe sono come lui, ma alcuni sono morti congelati nelle loro ceste, nelle quali vennero abbandonati sulle scale alla porta dei funzionari di San Pietroburgo, altri sono morti dalle educatrici, date loro dagli orfanotrofi affinché venissero cresciuti e nutriti\footnote{In quei tempi gli orfanotrofi davano ad delle educatrici alcuni bambini, affinché venissero nutriti, esse, gran parte delle quali veniva dalla Finlandia, ricevevano in cambio del loro lavoro denaro e vestiti. Come scrivevano i giornali del tempo, la mortalità di questi bambini era alta.}, altri sono morti ai seni rinsecchiti delle loro madri al tempo della \textit{fame di Samara}\footnote{1873-1874, dopo alcuni anni, in cui i campi diedero poco frutto, nella regione (\textit{óblast'}) di \textit{Samàra} ci fu  una terribile carestia.}, altri ancora sono morti nel fetore dei vagoni di terza classe, e tutti ora sono qui, tutti ora sono come angeli, tutti da Gesù, e lui sta in mezzo a loro, e tende loro le mani, e li benedice e  benedice le loro madri peccatrici...
	E le madri di questi bimbi stanno lì, in disparte, e piangono;
	ciascuna riconosce il proprio bimbo o bimba, e loro vengono volando e le baciano, con le proprie manine asciugano loro le lacrime e le pregano di non piangere, perché stanno così bene qui...
	
	Di sotto al mattino i portinai trovarono un piccolo cadavere di un bimbo accorso e morto per il freddo; cercarono anche sua madre...
	Lei era morta ancora prima di lui;
	entrambi si sono rivisti dal Signore Dio in cielo.
	
	\vspace{1.5em}
	
	E perché mai ho inventato una tale storia, che non si addice ad un comune e ragionevole diario, ancor meno a quello di uno scrittore?
	Io avevo promesso racconti che per la maggior parte narrano fatti realmente accaduti!
	Ed è proprio questo il punto, continuo ad avere l'impressione che tutto questo possa essere successo davvero, -- mi riferisco a ciò che accadde nel seminterrato e dietro al mucchio di legna, e per quanto riguarda l'albero di Natale di Gesù -- non lo so, come potrei dire, sarebbe potuto succedere, oppure no?
	Per questo sono un romanziere, per inventare.
	
\end{linenumbers}
\end{document}
